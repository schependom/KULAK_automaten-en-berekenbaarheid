% !TEX root = ../aanvullingen_cursus.tex
\documentclass[../aanvullingen_cursus.tex]{subfiles}

\begin{document}

De overgangsfunctie \[ \delta^* : Q \times \Sigma^* \to Q\] van een DFA, is een functie die een koppel \((q,s)\) afbeeldt op de unieke toestand \(p\) zodat \(q \pijl{s} p\). \\We kunnen deze functie ook inductief definiëren, en wel als volgt:
\begin{enumerate}
	\item \( \delta^*(q,\epsilon)=q \)
	\item \( \delta^*(q,aw) = \delta^*(\delta(q,a),w) \qquad \forall a\in\Sigma, w\in\Sigma^*\)
\end{enumerate}


\begin{stelling}
	In een DFA geldt dat \[\delta^*(q,wa) = \delta\left(\delta^*(q,w),a\right) \text{ voor } a\in\Sigma, w\in\Sigma^*\]
\end{stelling}


\begin{proof}

	We bewijzen dit per inductie op de lengte van \(w\).

	\begin{itemize}
		\item \underline{Basisstap}: ingeval de lengte van \(w\) gelijk is aan 0, geldt dat \(w=\epsilon\). In dat geval geldt dat
		\begin{align*}
			\delta^*(q,wa) &= \delta^*(q,\epsilon a) && w=\epsilon \\
							&= \delta^*(\delta(q,\epsilon),a) && \text{(2)} \\
							&= \delta^*(q,a) \\
							&= \delta(q,a) && \abs{a}=1 \\
							&= \delta\left(\delta^*(q,\epsilon),a\right) && \text{(1)} \\
							&= \delta\left(\delta^*(q,w),a\right) && w=\epsilon
		\end{align*}
		\item \underline{Inductiehypothese}: stel dat de stelling geldt voor strings \(w\) van hoogstens lengte \(\abs{w}=n\), m.a.w. dat voor zulke strings geldt dat \(\delta^*(q,wa) = \delta\left(\delta^*(q,w),a\right) \text{ voor } a\in\Sigma, w\in\Sigma_{\epsilon}^*\).
		\item \underline{Inductiestap}: we bewijzen dat de stelling ook geldt voor strings van lengte \(n+1\). Zo'n string \(w'\) kunnen we schrijven als \(w'=bw\) met \(\abs{w}=n,b\in \Sigma\). Nu geldt dat
		\begin{align*}
			\delta^*(q,w'a) &= \delta^*(q, bwa) && w'=bw \\
			&= \delta^*(\delta(q,b), wa) && \text{(1)} \\
			&= \delta(\delta^*(\delta(q,b),w), a) && \text{inductiehypothese} \\
			&= \delta\left(\delta^*(q,bw),a\right) && (2) \text{VRNL} \\
			&= \delta\left(\delta^*(q,w'),a\right) && w'=bw
		\end{align*}
	\end{itemize}

\end{proof}


\end{document}
\documentclass[../aanvullingen_cursus.tex]{subfiles}
\begin{document}


\begin{stelling}
	De taal \(L=\{ss \mid s\in \{a,b\}^*\}\) is niet contextvrij.
\end{stelling}


\begin{proof}
	We bewijzen dit aan de hand van het pompend lemma voor contextvrije talen.

	Stel dat er een pomplengte \(p\) bestaat zodanig dat elke string \(w \in L\) met lengte \(\abs{w}>p\) kan opgedeeld worden in 5 stukken \(u,v,x,y,z \in \Sigma^*\) zodanig dat \(w=uvxyz\) en zodat\begin{enumerate}
		\item \(\forall i \geq 0 : uv^ixy^iz \in L \)
		\item \(\abs{vy}>0\)
		\item \(\abs{vxy}\leq p\)
	\end{enumerate}

	Neem zo'n string \(w\) die (strikt) langer is dan \(p\), namelijk \begin{align*}
		w &= s_1s_2 && (s_1=s_2=a^pb^p \in \{a,b\}^*) \\
			&= a^pb^pa^pb^p \\
			&= \underbrace{\underbrace{a\cdot\cdots a}_p\underbrace{b\cdot\cdots b}_p}_{\abs{s_1}=2p} \underbrace{\underbrace{a\cdot\cdots a}_p\underbrace{b\cdot\cdots b}_p}_{\abs{s_2}=2p}
	\end{align*}
	Stel dat \(w=uvxyz\) en dat \(\abs{vy}>0\) en \(\abs{vxy}\leq p\). Dan bevat \(vxy\) hoogstens \(p\) symbolen en zijn er 3 mogelijke gevallen: \begin{enumerate}
		\item \underline{\(vxy\) zit volledig in \(s_1\)}: in dat geval worden er 1 of 2 symbolen uit \(s_1\) gepompt en geen enkel uit \(s_2\), wat wil zeggen dat het eerste deel van de resulterende strings \(uv^ixy^iz\) niet meer gelijk is aan het tweede deel en zulke strings dus onmogelijk kunnen behoren tot \(L\).
		\item \underline{\(vxy\) zit volledig in \(s_2\)}: dit verloopt volledig analoog aan het vorige geval.
		\item \underline{\(vxy\) zit deels in \(s_1\) en deels in \(s_2\)}: omdat de lengte hoogstens \(p\) is, wordt er ofwel een symbool \(b\) uit \(s_1\) gepompt, ofwel een symbool \(a\) uit \(s_2\), ofwel beiden. In alle 3 de gevallen zullen de gepompte strings niet van de vorm \(ss\) (met \(s \in \{a,b\}^*\)) zijn en dus kunnen de resulterende strings \(uv^ixy^iz\) onmogelijk tot de taal behoren.
	\end{enumerate}
	Gevolg: \(w\) kan niet gepompt worden en dus is \(L\) niet contextvrij.
\end{proof}


\end{document}
\documentclass[../aanvullingen_cursus.tex]{subfiles}
\begin{document}

\begin{stelling}
	Onderstaande constructie bewaart de taal, t.t.z. \( L_{\NFA_E}=L_E \)
	\begin{itemize}
		\item \( \NFA_{E_1E_2}=\operatorname*{concat}(\NFA_{E_1},\NFA_{E_2}) \)
		\item \( \NFA_{E_1^*}=\operatorname*{ster}(\NFA_{E_1}) \)
		\item \( \NFA_{E_1|E_2}=\operatorname*{unie}(\NFA_{E_1},\NFA_{E_2}) \)
	\end{itemize}
	Ofwel: talen bepaald door een reguliere expressie, worden herkend door een NFA (1. RE \(\rightarrow\) NFA)
\end{stelling}

\begin{proof}
	We bewijzen eerst volgende hulpstellingen:
	\begin{itemize}
		\item \underline{De concatenatie van \( \NFA_1 \) en \( \NFA_2 \) bepaalt \( L_{\NFA_1}L_{\NFA_2} \)}:

		We voeren volgende notatie in: \[ \text{NFA } C = \operatorname*{concat}(\NFA_{1},\NFA_{2}) \] Volgens de definitie van de concatenatie van twee talen geldt dat \[ L_{\NFA_1}L_{\NFA_2}=\{xy \mid x \in L_{\NFA_1}, y \in L_{\NFA_2}\} \] We moeten bewijzen dat \[ s \in L_{C} \Leftrightarrow s \in L_{\NFA_1}L_{\NFA_2} \]
		\begin{enumerate}
			\item[\( \Rightarrow \)] Neem aan dat \( s \in L_C \). Bij het parsen van deze string \( s \) met de machine C zullen we op een gegeven moment gegarandeerd in de toestand \(q_{f1}\) terechtkomen, aangezien dat de enige toestand is van waaruit we naar de machine \( \NFA_2 \) kunnen geraken. Dit gebeurt door een \epsilonboog  te nemen naar \( q_{s2} \). Noem de string die geparst is tijdens deze eerste fase \(x\) en neem de \epsilonboog van \( q_{f1} \) naar \( q_{s2} \). Er blijft -- vanuit deze starttoestand van \( \NFA_2 \) -- een string \( y \) over. Na het parsen van deze string \( y \) komen we in de toestand \( q_{f2} \) terecht, want \( s \in L_C \) en de enige (\(\varepsilon\)-)boog naar \( q_f \) vertrekt vanuit deze toestand. Omdat \( x \in L_{\NFA_1} \) (na het parsen van \( x \) belanden we in een aanvaardende toestand \( q_{f1} \) van \( \NFA_1 \)) en \( y \in L_{\NFA_2} \) (analoog), geldt dat \( s = xy \in L_{\NFA_1}L_{\NFA_2} \).
			\item[\(\Leftarrow\)] Als de string \( s \in L_{\NFA_1}L_{\NFA_2} \), dan bestaat \( s \) uit twee substrings, zodat \( s=xy \) met \( x \in L_{\NFA_1} \) en \( y \in L_{\NFA_2} \). Dat wil zeggen dat bij het doorlopen van C, we vanuit \( q_s \) in een eindig aantal stappen in \( q_{f1} \) terechtkomen. Van hieruit nemen we een \epsilonboog naar de begintoestand van \( \NFA_2 \). Vervolgens bereiken we na nog een eindig aantal extra stappen de toestand \(q_{f2}\), van waaruit we een \epsilonboog nemen naar de aanvaardende toestand \(q_f\). Hiermee hebben we aangetoond dat de string \(s\) wordt aanvaard door NFA \(C\), m.a.w. \(s \in L_{C}.\)
		\end{enumerate}

		\item \underline{De ster van \( \NFA_1 \) bepaalt \( L_{\NFA_1}^* \)}:

		We voeren volgende notatie in: \[ \text{NFA } S = \operatorname*{ster}(\NFA_{1}) \] De Kleene-ster van een taal \( L \) is de unie van alle talen \(L^n\) die ontstaan wanneer we deze taal \( n \) keer concateneren met zichzelf (\(n \in \N\)). Per definitie geldt dat \(\varepsilon \in L^*\), want er geldt dat \( L^0=\{\varepsilon\} \). We moeten bewijzen dat \[ s \in L_{S} \Leftrightarrow s \in L_{\NFA_1}^* \]

		Voor elk accepterend pad in \( L_S \), zijn de enige bogen die karakters uit \( \Sigma \) bevatten de bogen uit \(\NFA_1\). Bovendien: voor elke toestand \(q\) in \(\NFA_1\), gaat elk pad van deze toestand \(q\) naar de toestand \(q_f\) door \(q_{f1}\). Met andere woorden: de enige strings die in \(L_S\) zitten zijn \( \epsilon \) en \( x_1, x_2,x_3, ...\) (met \(x_i \in L_{\NFA_1} \)). Dit zijn precies die strings uit \( L_{\NFA_1}^* \).

		\item \underline{De unie van \( \NFA_1 \) en \( \NFA_2 \) bepaalt \( L_{\NFA_1} \cup L_{\NFA_2} \)}:

		We voeren volgende notatie in: \[ \text{NFA } U = \operatorname*{unie}(\NFA_{1},\NFA_{2}) \] Volgens de definitie van de unie van talen geldt dat \[ L_{\NFA_1} \cup L_{\NFA_2}=\{s \mid s \in L_{\NFA_1} \lor s \in L_{\NFA_2}\} \] We moeten bewijzen dat \[ s \in L_{U} \Leftrightarrow s \in L_{\NFA_1} \cup L_{\NFA_2} \]
		\begin{enumerate}
			\item[\( \Rightarrow \)] Neem aan dat \( s \in L_U \). Als we deze string parsen met de machine C, maken we in het begin de keuze om vanuit \( q_s \) de \epsilonboog te nemen naar ofwel \( q_{s1} \), ofwel \( q_{s2} \). We veronderstellen het eerste geval, namelijk de keuze voor de starttoestand van \(\NFA_1\), het andere geval verloopt analoog. Bij het parsen van de \( s \) belanden we uiteindelijk in \( q_f \), want dit is een aanvaarde string. Het bereiken van die toestand kan enkel met een \epsilonboog vanuit \( q_{f1} \) of \( q_{f2} \). Aangezien we in het begin gekozen hebben voor \( q_{s1} \) (en dus ook voor \( \NFA_1 \)), kan dat enkel vanuit \( q_{f1} \) gebeurd zijn. Het bereiken van \(q_{f1}\)\footnote{Hiermee wordt bedoeld dat de toestand bereikt wordt zonder dat er nog symbolen overschieten in \( s \) die nog geparst moeten worden.} wil precies zeggen dat \(s \in L_{\NFA_1}\) en dus ook \( s \in L_{\NFA_1} \cup L_{\NFA_2} \). We kunnen, zoals gezegd, hetzelfde aantonen voor de keuze van \( \NFA_2 \) in het begin.
			\item[\(\Leftarrow\)] Als de string \( s \in L_{\NFA_1} \cup L_{\NFA_2} \), dan geldt dat ofwel \( s \in L_{\NFA_1} \), ofwel \( s \in L_{\NFA_2} \). Veronderstel het eerste geval. Dan kunnen we bij het parsen van \( s \) aan de hand van de machine \( U \) de \epsilonboog naar \(q_{s1}\) nemen, waarna we de string \(s\) helemaal parsen, tot we in \(q_{f1}\) terechtkomen. Van hieruit kunnen we de \epsilonboog naar \(q_f\) nemen. We vinden dus dat \(s\) wordt aanvaard door \(U\) en dus dat \(s \in L_U\). Het andere geval (namelijk dat \(s \in L_{\NFA_2}\)) loopt nu volledig analoog.
		\end{enumerate}

	\end{itemize}

	We bewijzen nu de oorspronkelijke stelling aan de hand van structurele inductie:

	\begin{itemize}
		\item \underline{Basisstap}:
		We bewijzen dat de stelling geldt voor volgende basisgevallen:
		\begin{itemize}
			\item Als \( E=\varepsilon \), dan is \(L_E=\{\varepsilon\}\). Kijkend naar de constructie van de NFA voor dit basisgeval op pagina 21, zien we duidelijk dat deze NFA dezelfde taal bepaalt als \(E\) en dus geldt dat \(L_{\NFA_E}=L_E=\{\varepsilon\}\).
			\item Als \( E=\phi \), dan is \(L_E=\emptyset\). Kijkend naar de constructie van de NFA voor dit basisgeval op pagina 21, zien we duidelijk dat deze NFA dezelfde taal bepaalt als \(E\) en dus geldt dat \(L_{\NFA_E}=L_E=\emptyset\).
			\item Als \( E=a \in \Sigma \), dan is \(L_E=\{a\}\). Kijkend naar de constructie van de NFA voor dit basisgeval op pagina 21, zien we duidelijk dat deze NFA dezelfde taal bepaalt als \(E\) en dus geldt dat \(L_{\NFA_E}=L_E=\{a\}\).
		\end{itemize}
		\item \underline{Inductiestap}: neem aan dat de stelling geldt voor reguliere expressies \(E_1\) en \(E_2\): \[ L_{\NFA_{E_1}} = L_{E_1}, \quad L_{\NFA_{E_2}} = L_{E_2} \] Dan bewijzen we dat de stelling ook geldt (\( L_{\NFA_E}=L_E \)) voor de ster van \(E_1\), alsook voor de unie en concatenatie van beide RE's:
		\begin{itemize}
			\item \underline{Concatenatie}: De operatie wordt als volgt beschreven: \[ \NFA_{E_1E_2}=\operatorname*{concat}(\NFA_{E_1},\NFA_{E_2}) \] Uit bovenstaande hulpstelling voor de concatenatie volgt dat deze NFA de concatenatie bepaalt van de talen bepaald door \( \NFA_{E_1} \) en \( \NFA_{E_1} \). Verder gebruiken we ook de inductiehypothese: \[L_{\NFA_{E_1E_2}} \overset{\text{hulpstelling}}{=} L_{\NFA_{E_1}}L_{\NFA_{E_2}} \overset{\text{IH}}{=} L_{E_1}L_{E_2}\] Omdat volgens de definitie van \textit{de taal bepaald door een RE} geldt dat \(L_{E_1}L_{E_2}=L_{E_1E_2}\), volgt het te bewijzen nu direct: \( L_{\NFA_{E_1E_2}}=L_{E_1E_2} \)
			\item \underline{Ster}: Het operatie wordt als volgt beschreven: \[ \NFA_{E_1^*}=\operatorname*{ster}(\NFA_{E_1}) \] Uit bovenstaande hulpstelling voor de ster volgt dat deze NFA de taal \(L_{\NFA_{E_1}}^* \overset{\text{IH}}{=} L_{E_1}^*\) bepaalt. Dat wil precies zeggen dat \(L_{\NFA_{E_1^*}}=L_{E_1}^*=L_{E_1^*}\)\footnote{Gebruik hier ook de definitie van een taal bepaald door een reguliere expressie.}, zoals bewezen moest worden.
			\item \underline{Unie}: De operatie wordt als volgt beschreven: \[ \NFA_{E_1|E_2}=\operatorname*{unie}(\NFA_{E_1},\NFA_{E_2}) \] Uit bovenstaande hulpstelling voor de unie volgt dat deze NFA de taal \[L_{\NFA_{E_1}}\cup L_{\NFA_{E_2}} \overset{\text{IH}}{=} L_{E_1} \cup L_{E_2}\] bepaalt. Dat wil precies zeggen dat \(L_{\NFA_{E_1|E_2}}=L_{E_1} \cup L_{E_2}=L_{E_1 | E_2}\)\footnote{Idem}, zoals bewezen moest worden.
		\end{itemize}
	\end{itemize}

\end{proof}

\end{document}
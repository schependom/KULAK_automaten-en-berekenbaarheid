\documentclass[kulak]{kulakarticle}

\usepackage{amsmath}
\usepackage{amssymb}
\usepackage{amsfonts} % for \mathbb
\usepackage{amsthm}
\usepackage{tcolorbox}
\usepackage{mathtools}
\usepackage{siunitx}
\usepackage{subfiles}

\newcommand{\R}{\mathbb{R}} % Real numbers
\newcommand{\C}{\mathbb{C}} % Complex numbers
\newcommand{\Q}{\mathbb{Q}}
\newcommand{\N}{\mathbb{N}}
\newcommand{\powerset[1]}{\mathcal{P}(#1)}

\newcommand{\NFA}{\text{NFA}}
\newcommand{\DFA}{\text{DFA}}
\newcommand{\epsilonboog}{\( \varepsilon \)-boog }
\newcommand{\epsilonbogen}{\( \varepsilon \)-bogen}
\let\epsilon\varepsilon

\newcommand{\abs}[1]{\lvert #1 \rvert}
\newcommand{\pijl}[1]{\overset{#1}{\rightsquigarrow}}

\newcommand{\rood[1]}{\color{red}#1\color{black}}

\sisetup{output-decimal-marker={,}}
\sisetup{separate-uncertainty=true}		% Dit is voor de plus-minus
\sisetup{per-mode=fraction}

\DeclareSIUnit\biti{bit_i}
\DeclareSIUnit\combinatie{combinatie}

\usepackage[dutch]{babel}
\usepackage{hyperref}

\setlength{\parindent}{0pt}

\usepackage{amsthm}
\usepackage{tcolorbox}
\theoremstyle{definition}
\newtheorem*{stelling}{Stelling}

\tcolorboxenvironment{stelling}{
	boxrule=1pt,
	boxsep=2pt,
	left=2pt,right=2pt,top=2pt,bottom=2pt,
	sharp corners,
	before skip=\topsep,
	after skip=\topsep,
}

\title{Aanvullingen cursus A\&B}
\author{Vincent Van Schependom}
\date{Academiejaar 2024-2025}
\address{
	\textbf{Groep Wetenschap \& Technologie Kulak} \\
	Informatica \\
	Automaten \& Berekenbaarheid}

\begin{document}

	\maketitle

	\subsection*{Pagina 11, bewijs subalgebra:}

	\subfile{bewijzen/subalgebra}

	\subsection*{Pagina 15, zelf doen 4:}

	\subfile{allerlei/constructie_omgekeerde_taal}

	\subsection*{Pagina 18-20, de algebra van NFA's}

	\subfile{allerlei/constructie_concat_ster}

	\subsection*{Pagina 21, bewijs structurele inductie}

	\subfile{bewijzen/constructie_nfa_re_bewaart_taal}

	\subsection*{Pagina 26}

	\subfile{bewijzen/nfa_naar_gnfa}

	\subsection*{Pagina 28, bewijs DFA}

	\subfile{bewijzen/dfa_equivalent_nfa}

	\subsection*{Pagina 34, bewijs \( \DFA_{\text{min}} \)}

	\subfile{bewijzen/dfa_min_equivalent_dfa}


\end{document}




\documentclass[kulak]{kulakarticle}

\usepackage{amsmath}
\usepackage{amssymb}
\usepackage{amsfonts} % for \mathbb
\usepackage{amsthm}
\usepackage{tcolorbox}
\usepackage{mathtools}
\usepackage{siunitx}
\usepackage{subfiles}

\newcommand{\R}{\mathbb{R}} % Real numbers
\newcommand{\C}{\mathbb{C}} % Complex numbers
\newcommand{\Q}{\mathbb{Q}}
\newcommand{\N}{\mathbb{N}}
\newcommand{\powerset[1]}{\mathcal{P}(#1)}

\newcommand{\NFA}{\text{NFA}}
\newcommand{\DFA}{\text{DFA}}
\newcommand{\epsilonboog}{\( \varepsilon \)-boog }
\newcommand{\epsilonbogen}{\( \varepsilon \)-bogen}
\newcommand{\epsilonbogens}{\( \varepsilon \)-bogen }
\let\epsilon\varepsilon
\newcommand{\mnl}{MN\((L)\)}
\newcommand{\mnlm}{\text{MN}(L)}
\DeclareMathOperator{\reach}{reach}
\DeclareMathOperator{\opeen}{op_1}
\DeclareMathOperator{\optwee}{op_2}

\newcommand{\abs}[1]{\lvert #1 \rvert}
\newcommand{\enc}[1]{\langle #1 \rangle}
\newcommand{\pijl}[1]{\overset{#1}{\rightsquigarrow}}
\newcommand{\rpijl}[1]{\overset{#1}{\rightarrow}}

\newcommand{\rood[1]}{\color{red}#1\color{black}}

\sisetup{output-decimal-marker={,}}
\sisetup{separate-uncertainty=true}		% Dit is voor de plus-minus
\sisetup{per-mode=fraction}

\DeclareSIUnit\biti{bit_i}
\DeclareSIUnit\combinatie{combinatie}

\usepackage[dutch]{babel}
\usepackage{hyperref}

\setlength{\parindent}{0pt}

\usepackage{amsthm}
\usepackage{tcolorbox}
\theoremstyle{definition}
\newtheorem*{stelling}{Stelling}
\newtheorem*{definitie}{Definitie}

\tcolorboxenvironment{stelling}{
	boxrule=1pt,
	boxsep=2pt,
	left=2pt,right=2pt,top=2pt,bottom=2pt,
	sharp corners,
	before skip=\topsep,
	after skip=\topsep,
}
\tcolorboxenvironment{definitie}{
	boxrule=1pt,
	boxsep=2pt,
	left=2pt,right=2pt,top=2pt,bottom=2pt,
	sharp corners,
	before skip=\topsep,
	after skip=\topsep,
}

\title{Eerste gequoteerde oefenzitting}
\author{Automaten \& Berekenbaarheid}
\date{15 november 2024}
\address{
	\textbf{Groep Wetenschap \& Technologie Kulak} \\
	Bachelor Informatica \\
	Automaten \& Berekenbaarheid}

\begin{document}

	\maketitle

	\section*{Vraag 1}

	Beschouw de taal \( L = \{0^n1^m2^{n+m} \mid n,m \in \N \} \) over het alfabet \(\Sigma=\{0,1,2\}\).
	\begin{itemize}
		\item Is \(L\) regulier? Zo ja, geef dan een reguliere expressie \(E\) zodat \(L_E=L\). Zo niet, bewijs.
		\item Is \(L\) context-vrij? Zo ja, geef dan een context-vrije grammatica \(G\) zodat \(L_G=L\). Zo niet, bewijs.
	\end{itemize}

	\section*{Vraag 2}

	Voor een string \(s=s_1s_2\cdots s_n\) definiëren we VoegABToe\((s)\). Deze functie zal na elke letter in \(s\) afwisselend een \(a\) en een \(b\) zetten. Hieronder een aantal voorbeelden: \begin{align*}
		\text{VoegABToe}(\epsilon) &= \epsilon \\
		\text{VoegABToe}(a) &= aa \\
		\text{VoegABToe}(ccc) &= cacbca \\
		\text{VoegABToe}(appels) &= aapbpaeblasb
	\end{align*}

	Bewijs dat voor alle reguliere talen \(L\) geldt dat \[\text{ABTaal}(L) = \{ \text{voegABToe}(s) \mid s \in L \}\] ook regulier is.

	\section*{Vraag 3}

	Geef een context-vrije grammatica voor onderstaande talen:
	\begin{itemize}
		\item \(L_1 = \text{ABTaal}(\Sigma^*)\) met \(\Sigma = \{a,b\}\)
		\item \(L_2 = \{y \in \Sigma^* \mid \text{de lengte van $y$ is deelbaar door 2, door 3 of door beiden}\}\)
		\item \(L_3 = \{y \in \Sigma^* \mid \text{$aba$ is geen substring van } y \}\)
	\end{itemize}

	\section*{Vraag 4}

	Beschouw het alfabet \(\Sigma = \{a,b,c\}\). Voor elk symbool \(\sigma \in \Sigma\) en elke string \(s\in\Sigma^*\) beschouwen we de functie \(n_\sigma(s)\) die het aantal voorkomens van \(\sigma\) in de string \(s\) telt: \begin{align*}
		n_a(s) &= \text{aantal voorkomens van $a$ in $s$}\\
		n_b(s) &= \text{aantal voorkomens van $b$ in $s$}\\
		n_c(s) &= \text{aantal voorkomens van $c$ in $s$}
	\end{align*} Bewijs dat de volgende taal over \(\Sigma\) niet context-vrij is: \[L = \{s\in \Sigma^* \mid n_a(s) < n_b(s) \text{ en } n_b(s) < n_c(s)\}\]

\end{document}




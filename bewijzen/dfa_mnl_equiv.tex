\documentclass[../aanvullingen_cursus.tex]{subfiles}
\begin{document}


\begin{stelling}
	DFA's en MN(\(L\))-relaties zijn equivalent op isomorfisme na.
\end{stelling}


\begin{proof}
	We bewijzen twee richtingen:
	\begin{itemize}
		\item \underline{\textbf{Elke DFA bepaalt een MN(\(L\)) (equivalentie)relatie  op \(\Sigma^*\).}}

		Definieer voor elke toestand volgende deelverzameling van \(\Sigma^*\): \[\reach(q) = \{w \in \Sigma^* \mid \delta^*(q_s,w)=q\}\] De unie van al deze verzamelingen vormt een partitie van \(\Sigma^*\):
		\begin{itemize}
			\item \underline{Er bestaat geen \(\reach(q)=\emptyset\).} Elke string \(s \in \Sigma^*\) zit namelijk in een of andere \(\reach(q)\). De overgangsfunctie \(\delta\) van een DFA is totaal, dus bij het parsen van \(s\) kunnen voor elk symbool een boog volgen in de DFA, zodat we uiteindelijk in een of andere toestand terechtkomen. De string behoort dan precies tot de \(\reach(q)\) van deze toestand.
			\item \underline{De \(\reach(q)\)'s zijn disjunct.} Een string \(s \in \Sigma^*\) kan namelijk niet in twee \(\reach(q)\)'s zitten, want we hebben in een DFA nooit een keuze naar welke toestand we zullen overgaan: er zijn nooit twee verschillende bogen met eenzelfde symbool. Bij het parsen van \(s\) belanden we dus in een unieke toestand \(q\) en bijgevolg geldt dat \(s \in \reach(q)\).
			\item \underline{De unie van alle \(\reach(q)\)'s is precies \(\Sigma^*\).}
		\end{itemize}

		Omdat partities equivalentierelaties induceren en vice versa, kunnen we dus ook de geïnduceerde equivalentierelatie \(\sim_D\) beschouwen: \[x \sim_D y \quad \Leftrightarrow \quad x \text{ en } y \text{ behoren tot dezelfde} \reach(q) \quad \Leftrightarrow \quad \delta^*(q_s,x)=\delta^*(q_s,y)\] We tonen nu aan dat deze equivalentierelatie \(\sim_D\) een \mnl-relatie is. Herinner: \textbf{een equivalentierelatie \(\sim\) tussen strings is een Myhill-Nerode relatie voor \(L\) als \(\sim\) voldoet aan 3 voorwaarden}. We checken deze 3 voorwaarden nu voor \(\sim_D\):
		\begin{enumerate}
			\item \underline{De partitie is eindig}. Inderdaad: DFA's hebben een eindig aantal toestanden en bijgevolg zijn er dus ook een eindig aantal \(\reach(q)\)'s.
			\item \underline{Rechtscongruentie}: we willen aantonen dat \[x \sim_D y \quad \Rightarrow \quad xa \sim_D ya\] Stel dat \(x \sim_D y\). Dan geldt volgens de definitie van onze equivalentierelatie \(\sim_D\) dat beide strings behoren tot \(\reach(q)\) voor een \(q\in Q\), of nog dat \(\delta^*(q_s,x)=\delta^*(q_s,y)=q\). Vanuit deze toestand \(q\) hebben we voor elk symbool \(a \in \Sigma\) slechts één keuze met betrekking tot de boog die we nemen om over te gaan naar een nieuwe toestand. Noem deze nieuwe toestand \(q'=\delta(q,a)\). We hebben nu met de strings \(xa\) en \(ya\) dezelfde toestand \(q'\) bereikt, wat precies wil zeggen dat \(xa \sim_D ya\).
			\item \underline{\(\sim_D\) verfijnt de partitie \(\{L,\bar{L}\}\).} We willen aantonen dat \[x \sim_D y \quad \Rightarrow \quad (x\in L \Leftrightarrow y \in L)\] Stel dat \(x \sim_D y\). Als \(x \in L\), dan wil dat zeggen dat \(\delta^*(q_s,x)\in F\). Omdat \(x \sim_D y\) geldt dat \(\delta^*(q_s,x)=\delta^*(q_s,y)\) en dus geldt ook dat \(y \in L\). We kunnen de andere richting analoog bewijzen.
		\end{enumerate}

		\newpage
		\item \underline{\textbf{Elke \mnl-relatie \(\sim\) op \(\Sigma^*\) bepaalt een DFA}}:

		Gegeven een taal \(L \in L_\Sigma\). We construeren de DFA \((Q,\Sigma,\delta,q_s,F)\) als volgt:
		\begin{itemize}
			\item \(Q=\{x_{\sim} \mid x \in \Sigma^*\}\)
			\item \(q_s = \epsilon_{\sim}\)
			\item \(F = \{x_{\sim} \mid x \in L\}\)
			\item \(\delta(x_{\sim},a)=(xa)_{\sim}\)
		\end{itemize}

		Dit is inderdaad een DFA:
		\begin{itemize}
			\item  \underline{\(Q\) en \(F\) hebben slechts een eindig aantal toestanden}, omdat een \mnl-relatie geassocieerd is met een eindige partitie. Er zijn dus slechts een eindig aantal equivalentieklassen.
			\item \underline{De overgangsfunctie \(\delta\) is goed gedefinieerd}.

			Als \(y,z\in\Sigma^*\) tot dezelfde equivalentieklasse \(x_\sim\) behoren, dan behoren ze tot eenzelfde toestand \(q\in Q\). Na het volgen van een boog met een symbool \(a \in \Sigma\) vanuit deze toestanden, moeten we in de DFA voor beide strings in een eenzelfde nieuwe toestand \(q'\) terechtkomen, anders zouden er meerdere bogen met dat symbool \(a\) bestaan.

			We bewijzen dat we effectief in die toestand \(q'\) terechtkomen voor beide strings. Omdat volgens de \mnl-relatie op de strings in \(\Sigma^*\) de rechtscongruentie \(y \sim z \Rightarrow ya \sim za\) geldt, behoren de strings \(ya\) en \(za\) tot dezelfde equivalentieklasse \((xa)_\sim\). De definitie \(\delta(x_\sim,a)=(xa)_\sim\) is dus goed.
		\end{itemize}

		Tot slot bewijzen we nog dat de DFA de gegeven taal \(L \in L_\Sigma\) effectief bepaalt, of nog dat \(L_\DFA = L\): \[x \in L_\DFA \quad \overset{\Delta}{=} \quad \delta^*(\epsilon_\sim,x)\in F \Longleftrightarrow x_\sim \in F \quad \overset{\Delta}{=} \quad x\in L \] We bewijzen de overgang door per inductie op de lengte van \(x\) aan te tonen dat \[\delta^*(\epsilon_\sim,x)=x_\sim\]
		\begin{itemize}
			\item \underline{Basisstap:} als \(\abs{x}=0\), is \(x=\epsilon\) en geldt per definitie van \(\delta^*\) dat \(\delta^*(\epsilon_\sim,x)=\epsilon_\sim\)
			\item \underline{Inductiehypothese:} stel dat de stelling geldt voor strings \(x\) van lengte \(\abs{x}=n\)
			\item \underline{Inductiestap:} we bewijzen dat de stelling ook geldt voor strings van lengte \(n+1\). Zo'n string \(x'\) kunnen we schrijven als \(x'=xa\) met \(\abs{x}=n,a\in \Sigma\). Nu geldt dat
			\begin{align*}
				\delta^*(\epsilon_\sim,x')&=\delta^*(\epsilon_\sim,xa) && x'=xa\\
				&=\delta\left(\delta^*(\epsilon_\sim,x),a)\right) && \text{eigenschap } \delta^* \\
				&=\delta(x_\sim,a) && \delta^*(\epsilon_\sim,x) \overset{\text{inductiehypothese}}{=}x_\sim \\
				&=(xa)_\sim && \text{definitie } \delta \\
				&=(x')_\sim \quad \text{q.e.d.} && x'=xa
			\end{align*}
		\end{itemize}
	\end{itemize}
\end{proof}


\end{document}